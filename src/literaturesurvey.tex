\section{Literature survey}

The p2p storage networks have been actively researched and developed since
the beginning of 2000s. Their active development began with the growing
popularity of file-sharing services such as Napster \cite{napster}
and Gnutella \cite{gnutella}.
The \emph{Freenet} project, developed in 2000 was
a pioneer of distributed information sharing networks \cite{freenet}.
Freenet was the first distributed network which allowed anonymized
and censorship resistant publication and distribution of information.
% Freenet was not designed to be a long-term storage network, as
% the ``unpopular'' documents, which are not retrieved by the
% users frequently are eventually overwritten by a new content.
The network is fully operational nowadays.

Unlike Freenet, the \emph{PAST} distributed storage utility was a
research project (2001) focusing on scalability, high availability, persistence
and security \cite{past}. In many ways the basic ideas of PAST are
considered in our research. However the key difference is that PAST's
security and storage balancing approach is based on a usage of external
smartcards which are held by each PAST user and issued by a third
party broker. To the best of our knowledge PAST remained a research
project and is not available as an open-source or proprietary software.

Another field of p2p applications which have been successfully developed
and introduced in business is the distributed backup. The 

Like many other p2p applications which do not rely on a central party
for coordination, both Freenet and PAST use Distributed Hash Tables (DHT)
in order to locate network resources by their identifiers.

This study aims to research and develop a self-organized and fault-tolerance
distributed data storage network. Unlike the Freenet project \cite{freenet},
which has a strong focus on anonymous and censorship resistant \emph{information
distribution}, we rather focus on persistence of the network in a

  are focusing on data persistence in a  of each user's
% data uploaded to the network.

By-definition only the original user
or a predefined third party should be able to access the user's data
in our network. We are going to utilize the BitTorrent \cite{bittorrent-ma}
p2p file sharing protocol for data transfer.


Distributed peer-to-peer data storage networks have a great potential and have
been widely used since the beginning of 21st century
\cite{chord-01}.
