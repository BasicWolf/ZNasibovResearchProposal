\section{Literature survey}

The p2p storage networks have been actively researched and developed since
the beginning of 2000s. Their active development began with the growing
popularity of file-sharing services such as Napster \cite{napster}
and Gnutella \cite{gnutella}.

The \emph{Freenet} project, developed in 2000 was
a pioneer of distributed information sharing networks \cite{freenet}.
Freenet was the first distributed network which allowed anonymized
and censorship resistant publication and distribution of information.
% Freenet was not designed to be a long-term storage network, as
% the ``unpopular'' documents, which are not retrieved by the
% users frequently are eventually overwritten by a new content.
The network is fully operational nowadays.

Unlike Freenet, the \emph{PAST} distributed storage utility was a
research project (2001) focusing on scalability, high availability, persistence
and security \cite{past}. In many ways various ideas of PAST are
considered in our research. However the key difference is that PAST's
security and storage balancing approach is based on a usage of external
smartcards which are held by each PAST user and issued by a third
party broker.

The BitТorrent file-sharing protocol designed by Bram Cohen
in 2001 \cite{wiki:bittorrent} eventually became the most
popular protocol among the competitors.
Its advantage over plain direct download process over HTTP or FTP
is that when multiple downloads of the same file happen concurrently,
the downloaders upload to each other, making it possible for the file
source to support very large numbers of downloaders with only a modest
increase in its load \cite{bep3}.
As of June 2014, BitTorrent was responsible for 3.35\% of global
network traffic versus 1.32\% of FTP traffic, or 0.29\% of
Dropbox traffic \cite{paloalto-traffic}.

The original BitTorrent protocol relied on third-party \emph{tracker}
servers, in order to find other downloaders, called \emph{peers} and
join the distributed downloading process.
The protocol was lately extended with ability to locate a file by
its \emph{infohash} identifier via \emph{Distributed Hash Tables} (DHT).
A DHT is an analogue of a traditional hash table which offers increased
capacity and availability by partitioning the key space across a set
of cooperating peers and replicating stored data \cite{opendht}.
DHT techniques play a major role in a work of any p2p network
which does not rely on a centralized indexing party.
REWRITE THIS: They allow locating data and the peers which have it by querying
the table via a unique hash key obtained from the data.


USE:
OpenDHT: A Public DHT Service and Its Uses.pdf
Replication Strategies for Highly Available Peer-to-Peer Storage.pdf
Managing a Peer-to-Peer Data Storage System in a Selfish Society.pdf


Another field of p2p applications which have been successfully developed
and introduced in business is the distributed backup. The

Like many other p2p applications which do not rely on a central party
for coordination, both Freenet and PAST use Distributed Hash Tables (DHT)
in order to locate network resources by their identifiers.

This study aims to research and develop a self-organized and fault-tolerance
distributed data storage network. Unlike the Freenet project \cite{freenet},
which has a strong focus on anonymous and censorship resistant \emph{information
distribution}, we rather focus on persistence of the network in a

  are focusing on data persistence in a  of each user's
% data uploaded to the network.

By-definition only the original user
or a predefined third party should be able to access the user's data
in our network. We are going to utilize the BitTorrent \cite{bittorrent-ma}
p2p file sharing protocol for data transfer.


Distributed peer-to-peer data storage networks have a great potential and have
been widely used since the beginning of 21st century
\cite{chord-01}.
