\section{Literature survey}

Peer-to-peer file sharing networks have been actively researched and
developed since the beginning of 2000s. Their active development began
with the growing popularity of file-sharing services such as Napster
\citep{napster} and Gnutella \citep{gnutella}.

The \emph{Freenet} project developed by \citet{freenet}, was
a pioneer of distributed information sharing networks.
Freenet was the first distributed network which allowed anonymized
and censorship resistant publication and distribution of information.
Freenet was not designed to be a long-term storage network, as
the ``unpopular'' documents, which are not retrieved by the
users frequently are eventually overwritten by a new content.
Freenet is still being developed and is fully operational nowadays.

Unlike Freenet, the \emph{PAST} distributed storage utility
developed by \citet{past} was a research project focused
on scalability, high availability, persistence
and security. PAST introduced a novel approach to \pp network
security and storage balancing based on external smart cards.
The smart cards are meant to be unique, hold by each network
user and be issued by a third party.
PAST is available as a part of open-source FreePasty
\footnote{http://www.freepastry.org/FreePastry/} software package
and was successfully utilized in different \pp network researches.
\citet{p2p-intrusion} used PAST as a part of \pp collaborative
intrusion and malware detection network, while
\citet{p2p-social-security} used PAST as a part of distributed
social networks security evaluation study.

Although PAST can be utilized as a basic distributed storage
component, we state that its dependency on a third party smart
cards issuer which can restrict new users to join the network,
does not allow us to consider it as a genuine distributed network.
However there are important ideas of PAST and Freenet that could be
effectively adopted:
\begin{itemize}
\item \textbf{Layered design} - PAST's routing table as well
as resources (files) location functionality is based on Pastry
\citep{pastry} \emph{Dynamic Hash Table} (DHT) scheme. As stated by
\citet{past}, it is possible to layer PAST over other DHT schemes,
such as Chord \citep{chord-01} or Tapestry citep{tapestry}.
This feature allows isolating the important distributed resource
location part of a \pp network and carrying out a research on it
separately. We would discuss DHTs more thoroughly in the following
paragraphs.

\item \textbf{Caching} - In addition to replicating the files among
the network nodes, PAST utilizes the available free space for caching
the frequently requested data \citep{past-caching}. A similar approach
is used in Freenet \citep{freenet}. This allows decreasing the time
required to download a resource as well as the time required for
locating it.

\item \textbf{Data splitting} - In Freenet, a file is split and
uploaded to different nodes via chunks. \citet{dark-freenet} explain
this as a security measure against traffic analysis. As a positive
side effect, a full file can be retrieved from multiple nodes
at the same time. This results in a significant download speed
increase compared to downloading via a single connection.
\end{itemize}

We have briefly described the existing distributed storage projects
and some of their properties which we should consider in our research
as well as the properties which do not satisfy us. We will continue
with a description of the file transfer protocol of our choice.

The BitТorrent file-sharing protocol designed by Bram Cohen in 2001
citep{bittorrent-specs} eventually became the most popular protocol
among the competitors.
Its advantage over plain direct download process over HTTP or FTP
is that when multiple downloads of the same file happen concurrently,
the downloaders upload to each other, making it possible for the file
source to support very large numbers of downloaders with only a modest
increase in its load \citep{bittorrent-specs}.
As of June 2014, BitTorrent was responsible for 3.35\% of global
network traffic versus 1.32\% of FTP traffic, or 0.29\% of
Dropbox traffic \citep{paloalto:traffic}.

The original BitTorrent protocol relied on third-party \emph{tracker}
servers, in order to find other downloaders, called \emph{peers} and
join the distributed downloading process.
The protocol was lately extended with ability to locate a file by
its \emph{infohash} identifier via \emph{Distributed Hash Tables} (DHT).
A DHT is an analogue of a traditional hash table which offers increased
capacity and availability by partitioning the key space across a set
of cooperating peers and replicating stored data \cite{opendht}.
We consider to utilize the BitTorrent protocol for file transfers
and BitTorrent's \emph{Mainline DHT} \citep{bittorrent-dht} as the
initial DHT implementation in our network.

The further research of our network's DHT layer is required due to
different security questions which remain open.
\citet{dht-security-survey} thoroughly study different DHT attacks
and defense mechanisms against them. For example, any
DHT network is exposed to a \emph{Sybil} attack \citep{sybil}.
A Sybil attack is based on fact that any number of users can
connect to the DHT network, thus given a large number of
malicious nodes, the routing protocols can be compromised.
\citet{dht-security-survey} believe that only joining the network
via certificates issued by a trusted authority provides sufficient
defense against Sybil attacks. However \citet{dht-security-survey}
review distributed defense mechanisms, which provide a
certain protection against Sybil attack.
On of our major goals is to build a DHT network able to defend itself
against the known attacks. It is paramount, as the data location
mechanism depends on it.


USE:
OpenDHT: A Public DHT Service and Its Uses.pdf
Replication Strategies for Highly Available Peer-to-Peer Storage.pdf
Managing a Peer-to-Peer Data Storage System in a Selfish Society.pdf


Another field of p2p applications which have been successfully developed
and introduced in business is the distributed backup. The

Like many other p2p applications which do not rely on a central party
for coordination, both Freenet and PAST use Distributed Hash Tables (DHT)
in order to locate network resources by their identifiers.

This study aims to research and develop a self-organized and fault-tolerance
distributed data storage network. Unlike the Freenet project \cite{freenet},
which has a strong focus on anonymous and censorship resistant \emph{information
distribution}, we rather focus on persistence of the network in a

  are focusing on data persistence in a  of each user's
% data uploaded to the network.

By-definition only the original user
or a predefined third party should be able to access the user's data
in our network. We are going to utilize the BitTorrent \cite{bittorrent-ma}
p2p file sharing protocol for data transfer.


Distributed peer-to-peer data storage networks have a great potential and have
been widely used since the beginning of 21st century
\cite{chord-01}.
