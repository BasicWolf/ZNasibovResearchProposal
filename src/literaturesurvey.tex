\section{Literature survey}

Peer-to-peer file sharing networks have been actively researched and
developed since the beginning of 2000s. Their active development began
with the growing popularity of file-sharing services such as Napster
\citep{napster} and Gnutella \citep{gnutella}.

The \emph{Freenet} project developed by \citet{freenet}, was
a pioneer of distributed information sharing networks.
Freenet was the first distributed network that allowed anonymised
and censorship resistant publication and distribution of information.
% Freenet was not designed to be a long-term storage network, as
% the ``unpopular'' documents, which are not retrieved by the
% users frequently are eventually overwritten by a new content.
Freenet is still being developed and is fully operational.

Unlike Freenet, the \emph{PAST} distributed storage utility
developed by \citet{past} was a research project focused
on scalability, high availability, persistence
and security. PAST introduced a novel approach to \pp network
security and storage balancing based on external smart cards.
The smart cards are meant to be unique, retained by each network
user and be issued by a third party.
PAST is available as a part of open-source FreePasty
\footnote{http://www.freepastry.org/FreePastry/} software package
and was successfully utilized in different \pp network researches.
\citet{p2p-intrusion} used PAST as a part of \pp collaborative
intrusion and malware detection network, while
\citet{p2p-social-security} used PAST as a part of distributed
social network security evaluation study.

Although PAST can be utilized as a basic distributed storage
component, its dependency on a third party smart
cards issuer can restrict new users from joining the network.
Therefore, it cannot be considered as a genuine distributed network.
However, there are important ideas of PAST and Freenet, such as
\emph{data splitting}, \emph{layered design} and \emph{caching}
which could be adopted. They will be discussed in more detail in
the following paragraphs. 

% We have briefly described the existing distributed storage projects
% and some of their properties which we should consider in our research
% as well as the properties which do not satisfy us. We will continue
% with a description of the file transfer protocol of our choice.

In Freenet, a file is split and uploaded to different nodes via chunks.
\citet{dark-freenet} explain this as a security measure against traffic
analysis. As a positive side effect, a full file can be retrieved from
multiple nodes at the same time. This results in a significant download
speed increase compared to downloading via a single connection.
The \emph{BitТorrent} file-sharing protocol designed by Bram Cohen in 2001
\citep{bittorrent-specs} utilizes data splitting as its basic feature.
BitTorretn's advantage over plain direct download process like HTTP or FTP
is that when multiple downloads of the same file happen concurrently,
the downloaders upload to each other, making it possible for the file
source to support very large numbers of downloaders with only a modest
increase in its load \citep{bittorrent-specs}.
As of June 2014, BitTorrent was responsible for 3.35\% of global
network traffic versus 1.32\% of FTP traffic, or 0.29\% of
Dropbox traffic \citep{paloalto:traffic}.

The original BitTorrent protocol relied on third-party \emph{tracker}
servers, in order to find other downloaders, called \emph{peers} and
join the distributed downloading process.
The protocol was lately extended with ability to locate a file by
its \emph{infohash} identifier via \emph{Distributed Hash Tables} (DHT).
A DHT is an analogue of a traditional hash table which offers increased
capacity and availability by partitioning the key space across a set
of cooperating peers and replicating stored data \cite{opendht}.
We consider to utilize the BitTorrent protocol for file transfers
and BitTorrent's \emph{Mainline DHT} \citep{bittorrent-dht} as the
initial DHT implementation in our network.

PAST's routing table as well as resources (files) location functionality
is based on Pastry \citep{pastry} DHT scheme. As stated by
\citet{past}, it is possible to layer PAST over other DHT schemes,
such as \emph{Chord} \citep{chord-01} or \emph{Tapestry} \citep{tapestry}.
This feature allows isolating the DHT layer of the network and
carrying out a research on it separately. We should adopt this
approach in order to experiment with various DHT techniques without
modifying the storage logic of the network.

The further research of our network's DHT layer is required due to
different security questions which remain open.
A DHT security techniques survey by \citet{dht-security-survey}
describes various DHT attacks and defense mechanisms against them.
For example,
if a DHT network cannot guarantee that each of it's logical node
refers to a single physical entry, then it is exposed to a
generic \emph{Sybil} attack \citep{sybil}. One of the possible
attacks is that a large number of malicious nodes can corrupt the
honest nodes' routing tables.
\citet{dht-security-survey} believe that only joining the network
via certificates issued by a trusted authority provides sufficient
defense against Sybil attacks. However \citet{dht-security-survey}
review distributed defense mechanisms, which provide a certain
protection against Sybil attack.
On of our major goals is to build a DHT network able to defend itself
against the known attacks. It is paramount, as the data location
mechanism depends on it.

The last major research question is related to maximizing the data
availability once it was stored into the network.
\citet{past} outline the basic requirements for availability:

\begin{itemize}
\item A file remains available as long as one of the nodes that
      store the file is alive and reachable via the Internet.
\item The set of nodes that store the file is diverse in geographic
      location, administration, ownership, network connectivity, rule
      of law, etc.;
\item The number of files assigned to each node is balanced.
\end{itemize}

Although PAST relies on smart cards for storage quota control, it
implements an effective replication mechanism in case of local
storage shortage, which can be adopted in our network. In addition
to replicating the files among the network nodes, PAST utilizes the
available free space for caching the frequently requested data
\citep{past-caching}. A similar approach is used in Freenet
\citep{freenet}. This allows decreasing the time required to download
a resource as well as the time required for locating it.
\citet{past-caching} describe PAST storage management in-detail.
\citet{p2p-replication-strategy} state that splitting files into
blocks and utilizing \emph{erasure codes} (EC) results in the
most efficient replication strategy, compared to plain file blocks
or a whole file replication.
% They also state that a system using
% EC blocks can simply track the location of each peer holding
% \emph{any} block belonging to the object, so that an object can
% be reconstructed by requesting random blocks.

\citet{selfish-p2p} study the behaviour of \pp data storage system
users which behave selfishly. \citet{selfish-p2p} use the non-cooperative
game theory framework to answer the question of
``Whether it is socially better to
impose a scheme of limiting the users resources in the system
to their contribution level or to let a profit maximizing
monopoly set prices?''. This research shows that the users' behaviour
can be studied analytically. However the authors say that a complex
model would rather require a simulation.

% This study aims to research and develop a self-organized and fault-tolerance
% distributed data storage network. Unlike the Freenet project \cite{freenet},
% which has a strong focus on anonymous and censorship resistant \emph{information
% distribution}, we rather focus on persistence of the network in a

% By-definition only the original user
% or a predefined third party should be able to access the user's data
% in our network. We are going to utilize the BitTorrent \cite{bittorrent-ma}
% p2p file sharing protocol for data transfer.
