\section{The problem statement}

As discussed in introduction, storing the data in cloud services
can be very convenient for the end user, but it cannot be considered
secure. If at any point the service has an access to non-encrypted
user's data there is a chance that a 3d party, or the service itself
may intercept it for malicious actions.

One possible solution would encrypting the data prior to sending it
to a cloud service.

In this study we are aiming to research, prototype and build a p2p storage
network, focusing on the following properties:

\begin{itemize}
\item \textbf{Security} - all the user's data which is stored and transferred
  within the network is accessible only by its owner or a party with whom the
  owner is sharing an abstract secret only.
\item \textbf{Scalability} - a network can be established in various
  scales: from two-devices personal network up to an Internet-wide storage
  system.
\item \textbf{Self-organization} - a network is self-organized in such
  fashion that the users data is at least readable in case of it's spontaneous
  degradation, like availability of peers, the contributed storage size
  decrease, network lags and other failures.
\item \textbf{Decentralization} - a network is completely decentralized,
  with no key servers, so that there is no ``off'' button which shuts the
  whole network down.
\item \textbf{Usability} - the software should be easy and intuitive to
  use. Ideally the network client should be ready to use after a  simple
  setup with a minimum set of questions to the user.
\end{itemize}

Some of the properties described above have a well-known working solutions
available in various p2p protocols. But the described storage network brings
new challenges to solve:

\begin{itemize}
\item \textbf{Fairness} - the amount of the network resources available
  to each user should correlate with his or her contribution.
\end{itemize}

TODO:
* Why do we focus on p2p storage?
* What do we try to achieve?

* Fairness:
  * resource management
  * proof of storage and proof of transfer
  * how long the data may remain?
* Self-organization:
  * how the nodes should be grouped?
  * how the data should replicated?
  * node types and behaviour: mobile, stationary PC,
* Decentralization:
  * Defense against known generic attacks on p2p networks.
* Usability:
  * Access through web applications, i.e. browser?
  * Access on mobile platforms?
  *

% user \textbf{A} provides 1 gigabyte of storage
%   space for the network. User \textbf{B} provides 99 gigabytes of space
%   for the network. On average they contribute fifty gigabytes of space each,
%   but the difference is obvious. The fair access implies
