\section{The problem statement}

As discussed in the introduction, storing the data in cloud services
can be very convenient for the end user, yet it cannot be considered
secure. If at any point the service has access to a user's non-encrypted
data there is a chance that a third-party, or the service itself
could intercept it for malicious actions.

A user may encrypt data prior to uploading it to a cloud service,
which may affect usability.
For example one can utilize OpenPGP email-security software \citep{rfc3156} to
encrypt an email on the sender side, making in decipherable for
the end recipients only. However each recipient must possess a secret
key for decryption, which is usually not an option when contacting
someone for the first time.

In this research we aim to study, prototype and build a
cooperative decentralised \pp storage network, which is capable
of substituting cloud data storage services. In the proposed \pp network,
every user dedicates some of their storage space and bandwidth
to the needs of the other users. The network relies on
a well-known public cryptography mechanisms to ensure that
only the user and selected parties are able to access the data.

The following use cases are considered as a basis for the
network design and research questions:

\begin{itemize}
\item \textbf{Backup} - a user uploads files to the network and
      is able to retrieve them at a later date (for example,
      in a year's time).
\item \textbf{Synchronization} - a user takes some photos via a smartphone,
      which are uploaded to the network and then automatically
      downloaded to the user's PC.
\item \textbf{Sharing} - a user uploads a file to the network and
      shares the file identifier with a friend or other third party.
      The friend is then able to get the file from the network.
\end{itemize}

We consider the following features as mandatory in order to make
the network fully operational:

\begin{itemize}
\item \textbf{Decentralisation} - the network is completely decentralized,
      with no key servers, so that there is no ``off button'' which shuts
      the whole network down.
\item \textbf{Data Security} - all the user's data which is stored and transferred
      within the network is accessible only by its owner or a party to whom the
      owner is explicitly providing access.
\item \textbf{Routing security} - the routing and data location protocols
      provides security against known attack vectors.
\item \textbf{Availability} - a user's data remains readable,
      as long as a sufficient amount of peers who hold that data remain online.
\item \textbf{Network scalability} - the network can be established in various
      scales: from a two-devices personal network up to an Internet-based,
      large scale storage system.
\item \textbf{Storage balancing} - the network is able to balance the users'
      storage quota, so that the more storage space a user contributes,
      the more data they are able to store.
\item \textbf{Usability} - the software should be easy and intuitive to
      use. Ideally the network client should be ready to use after a
      simple setup with a minimum set of questions for the user.
\item \textbf{Commercialization} - The usage of the network is not limited
      by volunteer storage space and bandwidth sharing. A commercial company
      may utilize the user's resources in a private network for backup or any
      other data storing tasks and pay the users of the network for any
      provided storage.
\end{itemize}

We have to answer different research questions to implement these features
successfully. Because of the decentralised and open nature of the network,
each user who joins it for the first time,  will not be able to trust any
other user.

Our first research question is \emph{``how to organise the users connectivity
when none of them can be trusted, at least initially''}.
Our research will start with designing and implementing the computer model
of the network which will allow us to simulate users' activity as well as
\pp network attacks. The model will help to define the network's
self-organisation patterns which can maximise its security with minimal
computational overhead.

The same computer model will be used to answer our second research 
question, which is \emph{``how to organise the data replication to maximise
its availability''}. We will extend the initial computer model to simulate
the users file sharing activities 

The desired features provide a basis to the following research questions:


The ultimate problem of our research - is building the distributed
storage network described above. We have outlined the ready
components of the network that we may utilize, as well as the
components which need a further research in the following literature
survey.


% \begin{itemize}
% \item \textbf{Usage reporting} - the peers should be able to provide
%   a proof of storage and proof of data transfer. These values can be used
%   to balance the network resources between the peers, so that the amount
%   of the network resources available to each user should correlate with
%   his or her storage and bandwidth contribution. The questions to answer:
%   How data storage and transfer proof should be organized? How to protect
%   it from forgery?
%   % For example,
%   % user \textbf{A} provides 10 gigabyte of storage space for the network which
%   % is available 24 hours a day. User \textbf{B} provides 30 gigabytes of space
%   % for the network which is available 8 hours a day. Although the difference in
%   % contributed space is obvious, the uptime can be considered more valuable.
%   % Overall, the more resources a user provides to a network, the more resources
%   % are available for him.

% \item \textbf{Distributed data storage} - how should the data be replicated
%   among the peers? How long should it be kept by the peers if the network
%   is running out of space? Which properties of individual peer (i.e. uptime,
%   free space, bandwidth, device type etc.) should be  taken into consideration
%   when tranferring the data to it?

% \item \textbf{Network topology} - what is the optimal way of organizing the
%   network? How should the peers interconnect?

% \item \textbf{Network security} - how should the network defend against known
%   attacks on p2p networks?

% \item \textbf{Software and platforms} - is it possible to distribute the software
%   on various platforms (i.e. PC, mobile, browsers etc.), providing the same
%   functionality?
% \end{itemize}
