\section{The problem statement}

In this study we are aiming to research, prototype and build a volunteer
p2p storage network, with the focus on the following basic properties
and challenges to solve:

\begin{itemize}
\item \textbf{Security} - all the user's data stored in the network is
  accessible by the user or a party with whom a user is sharing an abstract
  ``secret'' only.
\item \textbf{Scalability} - a network can be established in various
  scales: from two-devices personal network up to an Internet-wide storage
  system.
\item \textbf{Self-organization} - a network is self-organized in such
  fashion that the users data is at least readable in case of it's spontaneous
  degradation, like availability of peers, the contributed storage size
  decrease, network lags and other failures.
\item \textbf{Decentralization} - a network is completely decentralized,
  with no key servers, so that there is no ``off'' button which shuts the
  whole network down.
\item \textbf{Usability} - the software should be easy and intuitive to
  use. Ideally the network client should be ready to use after a  simple
  ``setup'' with a minimum set of questions to the user.
\item \textbf{Fairness} - the amount of the network resources available
  to each user correlates with his or her contribution.
\end{itemize}


* Why do we focus on p2p storage?
* What do we try to achieve?

% user \textbf{A} provides 1 gigabyte of storage
%   space for the network. User \textbf{B} provides 99 gigabytes of space
%   for the network. On average they contribute fifty gigabytes of space each,
%   but the difference is obvious. The fair access implies