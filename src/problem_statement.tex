\section{The problem statement}

As discussed in introduction, storing the data in cloud services
can be very convenient for the end user, but it cannot be considered
secure. If at any point the service has an access to non-encrypted
user's data there is a chance that a third-party, or the service itself
may intercept it for malicious actions.

One possible solution would be encrypting the data prior to sending it
to a cloud service. Still, the points is that the cloud owns the data.
In this study we are aiming to research, prototype and build a p2p storage
network, which is capable of substituting a cloud data storage services.
focusing on the following properties:

\begin{itemize}
\item \textbf{Security} - all the user's data which is stored and transferred
  within the network is accessible only by its owner or a party with whom the
  owner is sharing an abstract secret only.
\item \textbf{Scalability} - a network can be established in various
  scales: from two-devices personal network up to an Internet-wide storage
  system.
\item \textbf{Self-organization} - a network is self-organized in such
  fashion that the users data is at least readable in case of it's spontaneous
  degradation, like availability of peers, the contributed storage size
  decrease, network lags and other failures.
\item \textbf{Decentralization} - a network is completely decentralized,
  with no key servers, so that there is no ``off'' button which shuts the
  whole network down.
\item \textbf{Usability} - the software should be easy and intuitive to
  use. Ideally the network client should be ready to use after a  simple
  setup with a minimum set of questions to the user.
\item \textbf{Commercialization} - The usage of the network is not limited
  by volunteer storage space and bandwidth sharing. A commercial company
  may utilize the user's resources in a private network for backup or any
  other data storing tasks and pay to the users of this network for provided
  storage.
\end{itemize}

Some of the properties described above have a well-known working solutions
available in various p2p protocols. But the described storage network brings
new challenges to solve:

\begin{itemize}
\item \textbf{Usage reporting} - the peers should be able to provide
  a proof of storage and proof of data transfer. These values can be used
  to balance the network resources between the peers, so that the amount
  of the network resources available to each user should correlate with
  his or her storage and bandwidth contribution. The questions to answer:
  How data storage and transfer proof should be organized? How to protect
  it from forgery?
  % For example,
  % user \textbf{A} provides 10 gigabyte of storage space for the network which
  % is available 24 hours a day. User \textbf{B} provides 30 gigabytes of space
  % for the network which is available 8 hours a day. Although the difference in
  % contributed space is obvious, the uptime can be considered more valuable.
  % Overall, the more resources a user provides to a network, the more resources
  % are available for him.

\item \textbf{Distributed data storage} - how should the data be replicated
  among the peers? How long should it be kept by the peers if the network
  is running out of space? Which properties of individual peer (i.e. uptime,
  free space, bandwidth, device type etc.) should be  taken into consideration
  when tranferring the data to it?

\item \textbf{Network topology} - what is the optimal way of organizing the
  network? How should the peers interconnect?

\item \textbf{Network security} - how should the network defend against known
  attacks on p2p networks?

\item \textbf{Software and platforms} - is it possible to distribute the software
  on various platforms (i.e. PC, mobile, browsers etc.), providing the same
  functionality?
\end{itemize}

