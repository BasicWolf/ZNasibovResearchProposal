\section{Introduction}

The Personal Computer (PC) has lost a great share of the market during the
last decade. Mobile devices and cloud computing have played a major role in
this course of events. Nevertheless, a mobile device like a smartphone or a
tablet is still considered a personal computing device in terms of data
storage and processing power. When we speak of cloud computing, we refer to
the services that store and process users' data remotely.
Surprisingly a lot of these services fall under this description,
for example: email providers such as
\emph{Gmail} (\url{https://gmail.com}),
image and file hosting services such as
\emph{Flickr} (\url{http://flickr.com}) and
\emph{Dropbox} (\url{http://dropbox.com}),
Voice over IP (VoIP) and Instant Messaging (IM) services, such as
\emph{Skype} (\url{http://skype.com}) and
\emph{WhatsApp} (\url{http://whatsapp.com}),
social networking services such as
\emph{Facebook} (\url{http://facebook.com}) and
\emph{LinkedIn} (\url{http://linkedin.com}).

Storing and processing users' data remotely has been a concern for
a lot of people around the globe. For example, the founder of GNU
Richard Stallman sees cloud comping as a trap aimed at forcing more
people to buy into locked, proprietary systems that would cost them
more and more over time \citep{stallman-cloud-08}.
Indeed a data stored in a cloud service is not controlled by the user.
A user has no idea where and how the data is stored, how well it is
protected and how it will be used by the service owner.
Google's \emph{terms of service} \citep{google-tos} and
\emph{privacy policy} \cite{google-privacy} state that a user's content
such as emails or location data via GPS coordinates is collected
and automatically analysed to
\emph{``provide personally relevant product
features, such as customized search results, tailored advertising,
and spam and malware detection''}.
These enforced terms leave a user
in awkward situation: either use the services and let all the information
to be collected or don't use the services at all.
The recent revelations
by Edward Snowden, a former US National Security Agency (NSA) employee
showed that the right to privacy of data can be severely ignored by a
legitimate government entity \citep{snowden-timeline}.
One of the documents obtained by the \emph{Guardian} newspaper,
revealed the existence of a government surveillance program
\emph{Prism}\cite{snowden-prism}.
Prism enabled the mass collection of seemingly private material
including search history, the content of emails, file
transfers and live chats, the document says. One of the ugly examples
of ignoring users' data privacy was the access of the contents of email,
prior to it being encrypted, on Microsoft's Outlook.com and Hotmail services.
Additionally access was granted to Microsoft's
cloud storage service \emph{SkyDrive}, provided by Microsoft to NSA
via Prism \citep{snowden-ms-nsa}.

With these facts in mind, why people still want to use cloud services?
Usability and availability are among the main factors that attract the users.
For example a Dropbox
user takes a photo via a smartphone, which is then automatically
uploaded to the cloud service and synchronised with the user's device(s).
The uploaded photo is also accessible via a standard web browser.

Another example is Google's free email service, Gmail.
Gmail provides access to email via
standard Internet Message Access (IMAP), and Simple Mail
Transfer (SMTP) protocols, thus making the mailbox accessible
via email client programs. However these features are available from
a wide range of competitors. What makes Gmail immensely popular
is its feature-rich and convenient web client and integration with
other Google services \citep{gmail-popularity}.

However, there are alternatives to proprietary cloud-based
Software-as-Service (SaaS).
For example, a personal server with e-mail,
File Transfer Protocol (FTP) and
Session Initiation Protocol (SIP)
services could theoretically substitute all the cloud services
mentioned above. In practice such a solution would be hard to maintain:
a personal server needs to be constantly online, and the user has to keep
the software up-to date, deal with security issues, hardware failures
etc.

To overcome such maintenance challenges an amount of
processor workload, network traffic routing and storage service load
can be distributed among the interconnected users.
The users of such a network are called peers and the network
itself is called a peer-to-peer (p2p) network.

One of the state of the art commercial \pp network examples is the
original Skype VoIP network protocol. It was designed to route the
voice traffic through a network of Skype users via
``the most effective path possible'', utilizing
users' bandwidth and CPU resources \citep{skype-p2p}.
Another relatively new and unorthodox p2p network example is
\emph{Bitcoin} (\url{http://bitcoin.org}). Bitcoin is
``the first decentralized peer-to-peer payment network that
is powered by its users with no central authority or middlemen''.
In a nutshell, the Bitcoin network is a storage and processing
mechanism for a chain of bitcoins transactions. Any user can
trace and verify the ownership and transfer of each virtual
coin from the moment it appeared in the network via
so-called bitcoin \emph{mining process}.
Finally, \emph{BitTorrent} (\url{http://bittorrent.org})-
a popular file sharing protocol, which is
rightly or wrongly associated with software piracy rather than
with information liberty. BitTorrent allows a piece of data to be
available, as long as any of the users, who have downloaded the original
chunk of data (from the original source, or any other user) remains
online. This leads to yet another brilliant property of BitTorrent:
given that there are always many users online, who share the same data,
the download can be distributed across different users (i.e. peers),
increasing the download speed up to the network channel bandwidth limits.

Peer-to-peer networks play a major role in Internet users' lives.
Yet, not all users, not even \pp software users are aware of them.
The reason of the appeal to the end users and \pp technologies
is the favourable development of modern computers and network channels.
With the growth of storage capabilities, processing power and network
bandwidth a typical user has gigaflops of processing power, gigabytes
of storage and gigabits of network bandwidth to spare. Combining all
these, it is possible to utilize the users' resources to create a
\pp network for storing and sharing data securely and independently
of any third-party service.
