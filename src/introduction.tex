\section{Introduction}

A Personal Computer (PC) has yielded a great share of marked during the
last decade. Mobile devices and cloud computing played a major role in
this course of events. Yet, a mobile device like a smartphone or a
tablet is still considered a personal computing device in terms of data
storage and processing. But when we speak of cloud computing, we refer to
the services that store and process users' data remotely.
Surprisingly a lot of various services fall under this description,
for example: email providers such as
\emph{Gmail} \footnote{\url{https://gmail.com}},
image and file hosting services such as
\emph{Flickr} \footnote{\url{http://flickr.com}} and
\emph{Dropbox} \footnote{\url{http://dropbox.com}},
Voice over IP (VoIP) and Instant Messaging (IM) services, such as
\emph{Skype} \footnote{\url{http://skype.com}} and
\emph{WhatsApp} \footnote{\url{http://whatsapp.com}},
social networking services such as
\emph{Facebook} \footnote{\url{http://facebook.com}} and
\emph{LinkedIn} \footnote{\url{http://linkedin.com}}.


Storing and processing users' data remotely has been a concern for
a lot of people around the globe. For example, the founder of GNU
Richard Stallman sees cloud comping as a trap\cite{stallman-cloud-08}.
Indeed a data stored in a cloud service is not controlled by the user.
A user has no idea where and how the data is stored, how well it is
protected and how it will be used by the service owner.
For example, \emph{Google} terms of service\cite{google-tos} and
privacy policy\cite{google-privacy} state that the a user's content
such as emails or a user's location like GPS coordinates is collected
and is automatically analyzed to ``provide personally relevant product
features, such as customized search results, tailored advertising,
and spam and malware detection''. These kind of terms leave a user
in awkward situation: either use the services and let all the information
to be collected or don't use the services at all.
Unfortunately the recent revelations\cite{snowden-timeline}
of Edward Snowden, a former US National Security Agency (NSA) employee
have revealed that a right for a privacy of the users' data can be
severely ignored by a legitimate government.
One of the documents obtained by the \emph{Guardian} magazine,
reveal an existence of a governmental surveillance program \emph{Prism}.
\cite{snowden-prism}. Prism allows officials to collect
material including search history, the content of emails, file
transfers and live chats, the document says. One of the ugly examples
of ignoring users' data privacy is a pre-encryption stage access
to email on Outlook.com, including Hotmail and an access to Microsoft's
cloud storage service \emph{SkyDrive}, all provided by Microsoft to NSA
via Prism \cite{snowden-ms-nsa}.

* Why people still use cloud services?
* What are the alternatives?
* Why do we focus on p2p storage?
* What do we try to achieve?
