\section{Introduction}

A Personal Computer (PC) has yielded a great share of marked during the
last decade. Mobile devices and cloud computing played a major role in
this course of events. Nevertheless, a mobile device like a smartphone or a
tablet is still considered a personal computing device in terms of data
storage and processing. When we speak of cloud computing, we refer to
the services that store and process users' data remotely.
Surprisingly a lot of various services fall under this description,
for example: email providers such as
\emph{Gmail}\footnote{\url{https://gmail.com}},
image and file hosting services such as
\emph{Flickr}\footnote{\url{http://flickr.com}} and
\emph{Dropbox}\footnote{\url{http://dropbox.com}},
Voice over IP (VoIP) and Instant Messaging (IM) services, such as
\emph{Skype}\footnote{\url{http://skype.com}} and
\emph{WhatsApp}\footnote{\url{http://whatsapp.com}},
social networking services such as
\emph{Facebook}\footnote{\url{http://facebook.com}} and
\emph{LinkedIn}\footnote{\url{http://linkedin.com}}.

Storing and processing users' data remotely has been a concern for
a lot of people around the globe. For example, the founder of GNU
Richard Stallman sees cloud comping as a trap\cite{stallman-cloud-08}.
Indeed a data stored in a cloud service is not controlled by the user.
A user has no idea where and how the data is stored, how well it is
protected and how it will be used by the service owner.
For example, \emph{Google} terms of service\cite{google-tos} and
privacy policy\cite{google-privacy} state that a content of a user
such as emails or a user's location like GPS coordinates is collected
and is automatically analyzed to
\emph{``provide personally relevant product
features, such as customized search results, tailored advertising,
and spam and malware detection''}.
These kind of terms leave a user
in awkward situation: either use the services and let all the information
to be collected or don't use the services at all.
The recent revelations\cite{snowden-timeline}
of Edward Snowden, a former US National Security Agency (NSA) employee
showed that a right for a privacy of the users' data can be
severely ignored by a legitimate government.
One of the documents obtained by the \emph{Guardian} magazine,
revealed an existence of a governmental surveillance program
\emph{Prism}\cite{snowden-prism}. Prism allows officials to collect
material including search history, the content of emails, file
transfers and live chats, the document says. One of the ugly examples
of ignoring users' data privacy is a pre-encryption stage access
to email on Outlook.com, including Hotmail and an access to Microsoft's
cloud storage service \emph{SkyDrive}, all provided by Microsoft to NSA
via Prism\cite{snowden-ms-nsa}.

Why people still use cloud services? Usability and availability are
among the main factors that attract the users. For example a Dropbox
user takes a photo via a smartphone camera, which is automatically
uploaded to a cloud if an internet connection is available.
The photo is automatically synchonized with the user's devices which
have Dropbox software installed on them. Moreover the user can
view and download the photo from any other device via a web browser.

Another example is Gmail, which provides an access to mail via
standard
POP\footnote{Post Office Protocol},
IMAP\footnote{Internet Message Access Protocol}
and SMTP\footnote{Simple Mail Transfer Protocol}
protocols, making the service accesible
via various email clients which are widely present on desktop
and mobile platforms. Still, Gmail's immense
popularity\cite{gmail-popularity}
do not come from these features, which are also available
from the competitors, but from its feature-rich and convenient
web client and integration with other Google services.

Is there an alternative to proprietary cloud-based
software-as-service?
Of course there is! A personal server with e-mail,
FTP\footnote{File Transfer Protocol} and
SIP\footnote{Session Initiation Protocol}
services could theoretically substitute all the cloud services
mentioned above. In practice such a solution would be hard to maintain:
a personal server need to be constantly online, a user has to keep
the software up-to date, deal with security issues, hardware failures
etc.

To overcome such maintainability difficulties an amount of
processor work, network traffic routing and storage service load
can be distributed among the interconnected users.
The users of such a network are called peers and the network
itself is called a peer-to-peer (p2p) network.

One of the state of the art commercial p2p network examples is the
original Skype VoIP network protocol. It was designed to route the
voice traffic through a network of Skype users via
``the most effective path possible''\cite{skype-p2p}, utilizing
users' bandwidth and CPU resources.
Another relatively new and unorthodox p2p network example is
\emph{Bitcoin}\footnote{http://bitcoin.org}. Bitcoin is
``the first decentralized peer-to-peer payment network that
is powered by its users with no central authority or middlemen''.
In a nutshell, the Bitcoin network is a storage and processing
mechanism for a chain of bitcoins transactions. Any user can
trace and verify the ownership and transfer of each virtual
coin from the moment it appeared in the network\footnote{Via
so-called bitcoin mining process}.
And finally BitTorrent - a popular file sharing protocol, which is
unfortunately widely associated with software piracy rather than
with information liberty. BitTorrent allows a piece of data to be
available, as long as any of the users, who has downloaded the original
chunk of data (from the original source, or any other user) remains
online. This leads to yet another brilliant property of BitTorrent:
given many users online, who share the same data, it can be downloaded
in pieces from different users (i.e. peers), increasing the download
speed up to the network channel bandwidth limits.

Peer-to-peer technologies play a major role in Internet users' lifes.
Yet, not all users, not even p2p software users are aware of them.
The reason of appealing for the end users and p2p technologies
is in a favourable development of modern computers and network channels.
With the growth of storage capabilities, processing power and network
bandwidth a typical user has gigaflops of processing power, gigabytes
of storage and megabits of network bandwidth to spare.

In this study we are aiming to research, prototype and build a p2p storage
network, with the focus on the following properties and challenges to
solve:

\begin{itemize}
\item \textbf{Security} - all the user's data stored in the network is
  accessible by the user or a party with whom a user is sharing an abstract
  ``secret'' only.
\item \textbf{Scalability} - a network can be established in various
  scales: from two-devices personal network up to an Internet-wide storage
  system.
\item \textbf{Self-organization} - a network is self-organized in such
  fashion that the users data is at least readable in case of it's spontaneous
  degradation, like availability of peers, the contributed storage size
  decrease, network lags etc.
\item \textbf{Usability} - the software should be easy and intuitive to
  use. Ideally the network client should be ready to use after a  simple
  ``setup'' with a minimum set of questions to the user.
\item \textbf{Fairness} - the amount of the network resources available
  to each user correlates with his or her contribution.

\end{itemize}



* Why do we focus on p2p storage?
* What do we try to achieve?

% user \textbf{A} provides 1 gigabyte of storage
%   space for the network. User \textbf{B} provides 99 gigabytes of space
%   for the network. On average they contribute fifty gigabytes of space each,
%   but the difference is obvious. The fair access implies